%-----------------------------------------------------------------------------------------------------------------------------------------------%
%	The MIT License (MIT)
%
%	Copyright (c) 2016 Jan Küster
%
%	Permission is hereby granted, free of charge, to any person obtaining a copy
%	of this software and associated documentation files (the "Software"), to deal
%	in the Software without restriction, including without limitation the rights
%	to use, copy, modify, merge, publish, distribute, sublicense, and/or sell
%	copies of the Software, and to permit persons to whom the Software is
%	furnished to do so, subject to the following conditions:
%	
%	THE SOFTWARE IS PROVIDED "AS IS", WITHOUT WARRANTY OF ANY KIND, EXPRESS OR
%	IMPLIED, INCLUDING BUT NOT LIMITED TO THE WARRANTIES OF MERCHANTABILITY,
%	FITNESS FOR A PARTICULAR PURPOSE AND NONINFRINGEMENT. IN NO EVENT SHALL THE
%	AUTHORS OR COPYRIGHT HOLDERS BE LIABLE FOR ANY CLAIM, DAMAGES OR OTHER
%	LIABILITY, WHETHER IN AN ACTION OF CONTRACT, TORT OR OTHERWISE, ARISING FROM,
%	OUT OF OR IN CONNECTION WITH THE SOFTWARE OR THE USE OR OTHER DEALINGS IN
%	THE SOFTWARE.
%
%-----------------------------------------------------------------------------------------------------------------------------------------------%
%----------------------------------------------------------------------------------------
%   TIMELINE HORIZONTAL CHART
%----------------------------------------------------------------------------------------
\newcounter{yearcount}
\newcounter{leftcount}
% env cvtimeline
%
% creates a vertical cv timeline
%
% param 1: start year
% param 2: end year
% param 3: overall width
% param 4: overall height             {\parbox{\myboxwidth}
\newenvironment{timelinehorizontal}[4]{
    \newcommand{\cvcategory}[2]{
        \node[label=\mbox{\colorbox{##1}{\strut\hspace{2pt}}\colorbox{white}{\textcolor{black}{##2}}}] at (0,-5) {}; % start year
    }
    \newcommand{\bxwidth}{4.5}
    \newcommand{\bxheight}{2}
    % creates a stretched box as cv entry headline followed by two paragraphs about 
    % the work you did
    % param 1:  event start month/year
    % param 2:  event end month/year
    % param 3:  event name
    % param 4:  level position y
    % param 5:  level position x
    % param 6:  color box
    % param 7:  color text
    % param 8:  level (position, use minus for left placement)
    \newcommand{\cvevent}[8] {
        \foreach \monthf/\yearf in {##1} {
            \foreach \montht/\yeart in {##2} {
                \pgfmathparse{-#3/\fullrange*((\yearf-#1)+(\monthf/12))}
                \let\startexp\pgfmathresult
                \pgfmathparse{-#3/\fullrange*((\yeart-#1)+(\montht/12))}
                \let\endexp\pgfmathresult
                \pgfmathparse{1/(\endexp-\startexp+1)}
                \let\lenexp\pgfmathresult
                \pgfmathparse{0.5*\endexp+0.5*\startexp}
                \let\midexp\pgfmathresult
                
                \pgfmathparse{-#3/\fullrange*((\yearf-#1)+(\monthf/12))}% get the event position
                \let\eventposition\pgfmathresult
                
                \filldraw[fill=##6, draw=none, opacity=0.9] (\startexp ,0.15+##8) rectangle (\endexp, 0.15+##8+0.5);
                
				\draw[draw=##6, line width=1.5pt, anchor=west] (\eventposition,0) -- (\eventposition+##5,##4);% line			
				
				\node[fill=##6,right=-0.1em,inner sep=0.5em, label={[label distance=0]0:\colorbox{##6}{\textcolor{##7}{##3}}}] at (\eventposition+##5,##4) {\textcolor{##7}{##1}};% textbox			
				
				% \node[below, inner sep=1.5em] at (\eventposition,0) {##1};%draw year below

				\draw[fill=white,draw=##6, line width=0.1em]  (\eventposition,0) circle (0.1);%year circle
            }
            \addtocounter{leftcount}{1}
        }
    }

    %--------------------------------------------------------------------------------------
    %   BEGIN
    %--------------------------------------------------------------------------------------

    \begin{tikzpicture}
    \setcounter{leftcount}{1}
    %calc fullrange= number of years
    \pgfmathparse{(#2-#1)}
    \let\fullrange\pgfmathresult
    \draw[draw=#4,line width=4pt] (0,0) -- (-#3,0) ;    %the timeline
    
    %for each year put a horizontal line in place
    \setcounter{yearcount}{1}
    \whiledo{\value{yearcount} < \fullrange}{
        \draw[fill=white,draw=#4, line width=2pt]  (-#3/\fullrange*\value{yearcount},0) circle (0.2);
        \stepcounter{yearcount}
    }
	
	%start year
    \filldraw[fill=white!100,draw=#4,line width=3pt] (0,0) circle (0.5);
    \node[draw=none, label=\textcolor{#4}{\textbf{\small#1}}] at (0,-0.35) {}; 
    %end year
    \filldraw[fill=white!100,draw=#4,line width=5pt] (-#3,0) circle (0.75);
    \node[draw=none, label=\textcolor{#4}{\textbf{\large#2}}] at (-#3,-0.35) {}; 
    
    
}%end begin part of newenv
{\end{tikzpicture}}