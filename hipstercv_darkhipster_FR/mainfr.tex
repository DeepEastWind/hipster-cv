\documentclass[darkhipster]{hipstercv}
% available options are: darkhipster, lighthipster, pastel, allblack, grey, verylight
\usepackage[utf8]{inputenc}
\usepackage[default]{raleway}
\usepackage[margin=1cm, a4paper]{geometry}

%----------------------------------------------------------------------------------------
% 	PIE CHART
%----------------------------------------------------------------------------------------
%-----------------------------------------------------------------------------------------------------------------------------------------------%
%	The MIT License (MIT)
%
%	Copyright (c) 2016 Jan Küster
%
%	Permission is hereby granted, free of charge, to any person obtaining a copy
%	of this software and associated documentation files (the "Software"), to deal
%	in the Software without restriction, including without limitation the rights
%	to use, copy, modify, merge, publish, distribute, sublicense, and/or sell
%	copies of the Software, and to permit persons to whom the Software is
%	furnished to do so, subject to the following conditions:
%	
%	THE SOFTWARE IS PROVIDED "AS IS", WITHOUT WARRANTY OF ANY KIND, EXPRESS OR
%	IMPLIED, INCLUDING BUT NOT LIMITED TO THE WARRANTIES OF MERCHANTABILITY,
%	FITNESS FOR A PARTICULAR PURPOSE AND NONINFRINGEMENT. IN NO EVENT SHALL THE
%	AUTHORS OR COPYRIGHT HOLDERS BE LIABLE FOR ANY CLAIM, DAMAGES OR OTHER
%	LIABILITY, WHETHER IN AN ACTION OF CONTRACT, TORT OR OTHERWISE, ARISING FROM,
%	OUT OF OR IN CONNECTION WITH THE SOFTWARE OR THE USE OR OTHER DEALINGS IN
%	THE SOFTWARE.
%
%-----------------------------------------------------------------------------------------------------------------------------------------------%

%counters for chart loop
\newcounter{a}
\newcounter{b}
\newcounter{c}

% draw a slice for a chart
% param 1: Circle form - 90 = quarter, 180 = half, 360 = full
% param 2: scale default=1 (scales only chart, not label text)
% param 3: border color
% param 4: label text color
% param 5: label bg color
% param 6:
\newenvironment{piechart}[5] {

	% draw a slice for a chart
	% param 1: value x of 100
	% param 2: label text
	% param 3: fill color
	% param 4:
	% param 5:
	% param 6:
	\newcommand{\slice}[3] {

		\setcounter{a}{\value{b}}
		\addtocounter{b}{##1}

		%set from angle point
		\pgfmathparse{\thea/100*#1}
	  	\let\pointa\pgfmathresult

		%set toanglepoint
		\pgfmathparse{\theb/100*#1}
	  	\let\pointb\pgfmathresult

		%set midangle
	 	\pgfmathparse{0.5*\pointa+0.5*\pointb}
	  	\let\midangle\pgfmathresult
		
		% draw the slice
	  	\filldraw[fill=##3!100,draw=#3!100, line width=2pt ] (0,0) -- (\pointa:#2) arc (\pointa:\pointb:#2) -- cycle;

	  	% draw label
	  	\node[label=\midangle:\colorbox{#5}{\textcolor{#4}{##2}}] at (\midangle:#2) {};

		\filldraw[fill=#3,draw=none] (0,0) circle (#2/2);
	}

	% execute commands
	\setcounter{a}{0}
	\setcounter{b}{0}
	\begin{tikzpicture}
}
{\end{tikzpicture}}

%------------------------------------------------------------------ Variablen



\newlength{\rightcolwidth}
\newlength{\leftcolwidth}
\setlength{\leftcolwidth}{0.3\textwidth}
\setlength{\rightcolwidth}{0.65\textwidth}

%------------------------------------------------------------------
\title{Hipster-CV}
\author{\LaTeX{} Ninja}
\date{March 2019}

\pagestyle{empty}
\begin{document}


\thispagestyle{empty}
%-------------------------------------------------------------

\section*{Start}



\header{\bgupper{cvyellow}{headerfontbox}{\bfseries\Huge Achraf Najmi}}{\bg{cvyellow}{headerfontbox}{\large Électro Info Physicien}}{\bg{cvyellow}{headerfontbox}{
\begin{minipage}[t]{0.5\textwidth}
Je suis diplômé en Physique Électronique avec une expérience de travail dans des projets de recherche. \\
Je suis toujours à la recherche de nouveaux projets interdisciplinaires passionnants liés à la physique.
\end{minipage}}}{./pic/me.JPG}{headerblue}{3.5cm}{1cm}



%------------------------------------------------

% hier muss die "unsichtbare" Überschrift rein, weil er sonst nicht die Paracols startet... komisch...
\subsection*{}
\vspace{4em}

\setlength{\columnsep}{1.5cm}
\columnratio{0.3}[0.65]
\begin{paracol}{2}
\hbadness5000
%\backgroundcolor{c[1]}[rgb]{1,1,0.8} % cream yellow for column-1 %\backgroundcolor{g}[rgb]{0.8,1,1} % \backgroundcolor{l}[rgb]{0,0,0.7} % dark blue for left margin

\paracolbackgroundoptions

% 0.9,0.9,0.9 -- 0.8,0.8,0.8

\vspace{-2em}
\footnotesize
{\setasidefontcolour
\bgupper{cvyellow}{iconcolour}{Faits} \\
\bg{cvyellow}{iconcolour}{Personnel} \\

\begin{tabular}{ll | l}
\faMale & Nom Complet & Achraf NAJMI \\
\faBirthdayCake & Né le &  24/07/1994 \\
% \faGlobe & Nationality & Moroccan  \\
\faMapMarker & Emplacement & Casablanca, Maroc \\
\end{tabular}

\bigskip

\bg{cvyellow}{iconcolour}{Physique Électronique} \\

\begin{tabular}{l | l}
\faRobot \hspace{0.1em} Électronique & \faChargingStation \hspace{0.1em} Électrotechnique  \\
\faLaptop \hspace{0.1em} Numérique & \faBroadcastTower \hspace{0.1em} Télécommunication \\
\faSitemap  \hspace{0.1em} Réseau
\end{tabular}

\bigskip

\bgupper{cvyellow}{iconcolour}{Capacites} \\
\bg{cvyellow}{iconcolour}{Langues}
\bigskip


\begin{minipage}[t]{\leftcolwidth}
\begin{tabular}{ll | l}
\faLanguage & \textbf{Français} & B2 \pictofraction{\faCircle}{cvpurple}{4}{black!30}{2}{\tiny}\\
\faLanguage & \textbf{Anglais} & A2 \pictofraction{\faCircle}{cvpurple}{2}{black!30}{4}{\tiny}
\end{tabular}
\end{minipage}

\bigskip

\bg{cvyellow}{iconcolour}{Informatique et Programmation}\\

\bubblediagram{{\textbf{\faPython \hspace{0.1em} Python 3}}, \faFileCode \hspace{0.1em} pyinstaller, \faSuperscript \hspace{0.1em} sympy, \faSortNumericDown \hspace{0.1em} numpy, \faTable \hspace{0.1em} pandas, \faFeather \hspace{0.1em} tkinter, \faChartLine \hspace{0.1em} matplotlib}

% \bigskip
\smallskip

\bg{cvyellow}{iconcolour}{Environnement}\\

\color{labelcolour}{Syntaxe:}
% \hspace{0.2em} 
\icon{\faPython}{labelcolour}{\Large} 
\color{labelcolour}{3.7 \& 3.8 \hspace{0.7em} OS:}
% \hspace{0.3em} 
\icon{\faWindows}{labelcolour}{\Large} 
% \hspace{0.3em} 
\icon{\faLinux}{labelcolour}{\Large} 

\bigskip
% \smallskip

\bg{cvyellow}{iconcolour}{Programmes Logiciels} \\

\begin{minipage}[t]{0.3\textwidth}
\begin{tabular}{r @{\hspace{0.5em}}l}
     \bg{skilllabelcolour}{iconcolour}{PyCharm IDE} & \barrule{0.4}{0.5em}{cvgreen} \\
     % \bg{skilllabelcolour}{iconcolour}{Microsoft Office} & \barrule{0.4}{0.5em}{orange} \\
     \bg{skilllabelcolour}{iconcolour}{Visual Studio IDE} & \barrule{0.35}{0.5em}{cvpurple}\\
     \bg{skilllabelcolour}{iconcolour}{Proteus CAD} & \barrule{0.3}{0.5em}{cvblue} \\
     \bg{skilllabelcolour}{iconcolour}{\faMarkdown \hspace{0.1em} Markdown} & \barrule{0.25}{0.5em}{orange} \\
     \bg{skilllabelcolour}{iconcolour}{\LaTeX} & \barrule{0.2}{0.5em}{cvyellow} \\
     \bg{skilllabelcolour}{iconcolour}{\faCodeBranch \hspace{0.1em} \faGit} & \barrule{0.15}{0.5em}{red} \\
\end{tabular}


% \dashrule{}{}
\end{minipage}

\smallskip
\bg{cvyellow}{iconcolour}{Technologies}\\
% \parbox[b][110pt][c]{0.30\textwidth}{
% PIE CHART	
\begin{piechart}{360}{1}{darkgray}{iconcolour}{skilllabelcolour}
			\slice{18}{Projets}{green}
			\slice{29}{Développement}{headerblue}
			\slice{29}{Maintenance}{cvyellow}
			\slice{14}{Recherche}{orange}
			\slice{10}{Désigne}{cvpurple}
\end{piechart}\vspace{-4em}
% }
\phantom{turn the page}

\phantom{turn the page}
}
%-----------------------------------------------------------

\switchcolumn

\small
\vspace{-2em}
\section*{CV Court}

\begin{tabular}{r| p{0.4\textwidth} c}
    \cvevent{2019--2021}{Développement Informatique}{Casablanca}{Maroc \color{cvred}}{Langage Python: Syntaxe, GUI, Data Science}{./pic/python.png} \\
    
    \cvevent{2013--2021}{Maintenance Informatique FreeLance}{Casablanca}{Maroc \color{cvred}}{Diagnostique et réparation de Hardware \& Software...}{./pic/computer-maintenance.png} \\
    
    \cvevent{2012--2019}{Études Universitaires}{FSBM · UHIIC · Casablanca}{Maroc\color{cvred}}{Physique Électronique}{./pic/univh2fsbm.png}
\end{tabular}

\vspace{-1em}

% \begin{minipage}[t]{0.4\textwidth}
\section*{Diplômes}
\begin{tabular}{r| p{0.5\textwidth} c}
    \cvdegree{2019}{Licence d'Études Fondamentales - LEF}{Université Hassan II de Casablanca}{FSBM \color{headerblue}}{Filière: Science Matière Physique | Parcours : Électronique}{./pic/univh2fsbm.png} \\
    
    \cvdegree{2017}{Diplôme d'Études Universitaires Générales - DEUG}{Université Hassan II de Casablanca}{FSBM \color{headerblue}}{Filière: Science Matière Physique}{./pic/univh2fsbm.png} \\
    
    \cvdegree{2012}{Diplôme du Baccalauréat}{Lycée}{Ibno Zaidone \color{headerblue}}{Série : Sciences Expérimentales | Option : Science de la Vie et la Terre}{./pic/minister.jpg}
\end{tabular}
% \end{minipage}\hfill

\vspace{-1em}
\section*{Projets}
\begin{tabular}{r| p{0.5\textwidth} c}
    \cvdegree{2020}{Projet de Hydrogéologie}{PyCharm Community Edition}{Python 3.7\color{cvorange}}{\href{https://github.com/DeepEastWind/Hydrogeologie}{\icon{\faGithub}{cvpurple}{}github.com/DeepEastWind/Hydrogeologie} (PROJET PRIVÉ) Hydrologie des eaux souterraines Livre de David Keith Todd}{./pic/earth.png} \\
    
    \cvdegree{2020}{Projet de MathPy}{PyCharm Community Edition}{Python 3.7\color{cvorange}}{\href{https://www.github.com/DeepEastWind/MathPy}{\icon{\faGithub}{cvpurple}{}github.com/DeepEastWind/MathPy} (PROJET PRIVÉ) Calculatrice scientifique avec des fonctionnalités modernes}{./pic/MathPy.png} \\
    
    \cvdegree{2018}{Projet de Fin d'Études}{UHIIC}{Faculté des Sciences Ben M'sick \color{cvorange}}{Conception et simulation d'un système photovoltaïque et réalisation d'un suiveur solaire.}{./pic/univh2fsbm.png} \\
    
    % \cvdegree{2018}{Projet Entreprenariat}{UHIIC}{Faculté des Sciences Ben M'sick \color{cvorange}}{Projet de Création une Entreprise Commerciale "electrocherry".}{./pic/electro.png}
\end{tabular}

\begin{minipage}[t]{0.3\textwidth}

\section*{Info et Program}
\textcolor{iconcolour}{Actuellement, je développe des applications en temps réel avec Python, et crée une interface graphique avec tkinter, et extrait du script `file.py` un fichier exécutable par pyinstaller en utilisant auto py to exe.}

\end{minipage}\hfill
\begin{minipage}[t]{0.3\textwidth}

\section*{Programmes Logiciels}
\textcolor{iconcolour}{J'utilise régulièrement pour mes projets de développement PyCharm Community et Visual Studio comme IDE, et j'utilise pour la capture schématique et la simulation Proteus CAD.}

\end{minipage}

\begin{minipage}[t]{0.65\textwidth}
\vspace{0.5em}
\section*{Activites}
% usage \hobbyicon{<fontawesome icon}{Text}{background color of circle}{size of icon}{space text below icon}
\mbox{
	\parbox[b][2cm][c]{7.5cm}{
		\hobbyicon{\color{cvyellow}\faPlayCircle}{Tutoriels \\}{headerblue}{\iconsize}{2em}
		% \hspace{1em}
		\hobbyicon{\color{iconcolour}\faNewspaper}{Actualités \\Développement}{cvgreen}{\iconsize}{2em} % \hfill
		% \hspace{1em}
		\hobbyicon{\color{labelcolour}\faChess}{Brain \\Games}{cvpurple}{\iconsize}{2em}
		\hspace{1em}
		\hobbyicon{\color{iconcolour}\faBicycle}{Sport \\}{cvorange}{\iconsize}{2em}
		% \hspace{1em}
		% \hobbyicon{\color{iconcolour}\faBicycle}{Fitness\\ \& GYM}{pink}{\iconsize}{2em}\\
		}
	\parbox[b][2cm][c]{5cm}{
			\textcolor{iconcolour}{Parmis les dominantes activités que je pratique est de naviguer sur internet à la recherche de nouvelles sur le monde d'informatique et  entraîné sur tous les tutoriels qui m'intéresse.}
		}
	}
%\hobbyicon{\color{iconcolour}\faFutbol}{Playing}{red}{\iconsize}{2em}
\end{minipage}









\vfill{} % Whitespace before final footer

%----------------------------------------------------------------------------------------
%	FINAL FOOTER
%----------------------------------------------------------------------------------------
\setlength{\parindent}{0pt}
\begin{minipage}[t]{\rightcolwidth}
\begin{center}\fontfamily{\sfdefault}\selectfont \color{black!70}
{\href{callto:+212622902221}{\icon{\faMobile}{black}{} +212 6 22 90 22 21} \href{mailto:najmi.achraf@gmail.com}{\icon{\faAt}{red}{} najmi.achraf@gmail.com} \href{https://www.github.com/DeepEastWind}{\icon{\faGithub}{cvpurple}{} github.com/DeepEastWind}  \newline\href{https://www.linkedin.com/in/achraf-d-najmi}{\icon{\faLinkedin}{blue}{} linkedin.com/in/achraf-d-najmi}\href{https://stackoverflow.com/users/12854948}{\icon{\faStackOverflow}{iconcolour}{} stackoverflow.com/users/12854948}
}
\end{center}\vspace{-4em}
\end{minipage}


\end{paracol}
\end{document}
