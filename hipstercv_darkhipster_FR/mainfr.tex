\documentclass[darkpython]{hipstercv}
% available options are: darkpython, lightpython, darkhipster, lighthipster, pastel, allblack, grey, verylight
\usepackage[utf8]{inputenc}
\usepackage[default]{raleway}
\usepackage[margin=1cm, a4paper]{geometry}

%----------------------------------------------------------------------------------------
% 	PIE CHART
%----------------------------------------------------------------------------------------
%-----------------------------------------------------------------------------------------------------------------------------------------------%
%	The MIT License (MIT)
%
%	Copyright (c) 2016 Jan Küster
%
%	Permission is hereby granted, free of charge, to any person obtaining a copy
%	of this software and associated documentation files (the "Software"), to deal
%	in the Software without restriction, including without limitation the rights
%	to use, copy, modify, merge, publish, distribute, sublicense, and/or sell
%	copies of the Software, and to permit persons to whom the Software is
%	furnished to do so, subject to the following conditions:
%	
%	THE SOFTWARE IS PROVIDED "AS IS", WITHOUT WARRANTY OF ANY KIND, EXPRESS OR
%	IMPLIED, INCLUDING BUT NOT LIMITED TO THE WARRANTIES OF MERCHANTABILITY,
%	FITNESS FOR A PARTICULAR PURPOSE AND NONINFRINGEMENT. IN NO EVENT SHALL THE
%	AUTHORS OR COPYRIGHT HOLDERS BE LIABLE FOR ANY CLAIM, DAMAGES OR OTHER
%	LIABILITY, WHETHER IN AN ACTION OF CONTRACT, TORT OR OTHERWISE, ARISING FROM,
%	OUT OF OR IN CONNECTION WITH THE SOFTWARE OR THE USE OR OTHER DEALINGS IN
%	THE SOFTWARE.
%
%-----------------------------------------------------------------------------------------------------------------------------------------------%

%counters for chart loop
\newcounter{a}
\newcounter{b}
\newcounter{c}

% draw a slice for a chart
% param 1: Circle form - 90 = quarter, 180 = half, 360 = full
% param 2: scale default=1 (scales only chart, not label text)
% param 3: border color
% param 4: label text color
% param 5: label bg color
% param 6:
\newenvironment{piechart}[5] {

	% draw a slice for a chart
	% param 1: value x of 100
	% param 2: label text
	% param 3: fill color
	% param 4:
	% param 5:
	% param 6:
	\newcommand{\slice}[3] {

		\setcounter{a}{\value{b}}
		\addtocounter{b}{##1}

		%set from angle point
		\pgfmathparse{\thea/100*#1}
	  	\let\pointa\pgfmathresult

		%set toanglepoint
		\pgfmathparse{\theb/100*#1}
	  	\let\pointb\pgfmathresult

		%set midangle
	 	\pgfmathparse{0.5*\pointa+0.5*\pointb}
	  	\let\midangle\pgfmathresult
		
		% draw the slice
	  	\filldraw[fill=##3!100,draw=#3!100, line width=2pt ] (0,0) -- (\pointa:#2) arc (\pointa:\pointb:#2) -- cycle;

	  	% draw label
	  	\node[label=\midangle:\colorbox{#5}{\textcolor{#4}{##2}}] at (\midangle:#2) {};

		\filldraw[fill=#3,draw=none] (0,0) circle (#2/2);
	}

	% execute commands
	\setcounter{a}{0}
	\setcounter{b}{0}
	\begin{tikzpicture}
}
{\end{tikzpicture}}


%----------------------------------------------------------------------------------------
% 	TIMELINE VERTICAL & HORIZONTAL CHART
%----------------------------------------------------------------------------------------
%-----------------------------------------------------------------------------------------------------------------------------------------------%
%	The MIT License (MIT)
%
%	Copyright (c) 2016 Jan Küster
%
%	Permission is hereby granted, free of charge, to any person obtaining a copy
%	of this software and associated documentation files (the "Software"), to deal
%	in the Software without restriction, including without limitation the rights
%	to use, copy, modify, merge, publish, distribute, sublicense, and/or sell
%	copies of the Software, and to permit persons to whom the Software is
%	furnished to do so, subject to the following conditions:
%	
%	THE SOFTWARE IS PROVIDED "AS IS", WITHOUT WARRANTY OF ANY KIND, EXPRESS OR
%	IMPLIED, INCLUDING BUT NOT LIMITED TO THE WARRANTIES OF MERCHANTABILITY,
%	FITNESS FOR A PARTICULAR PURPOSE AND NONINFRINGEMENT. IN NO EVENT SHALL THE
%	AUTHORS OR COPYRIGHT HOLDERS BE LIABLE FOR ANY CLAIM, DAMAGES OR OTHER
%	LIABILITY, WHETHER IN AN ACTION OF CONTRACT, TORT OR OTHERWISE, ARISING FROM,
%	OUT OF OR IN CONNECTION WITH THE SOFTWARE OR THE USE OR OTHER DEALINGS IN
%	THE SOFTWARE.
%
%-----------------------------------------------------------------------------------------------------------------------------------------------%
%----------------------------------------------------------------------------------------
%   TIMELINE HORIZONTAL CHART
%----------------------------------------------------------------------------------------
\newcounter{yearcount}
\newcounter{leftcount}
% env cvtimeline
%
% creates a vertical cv timeline
%
% param 1: start year
% param 2: end year
% param 3: overall width
% param 4: overall height             {\parbox{\myboxwidth}
\newenvironment{timelinehorizontal}[4]{
    \newcommand{\cvcategory}[2]{
        \node[label=\mbox{\colorbox{##1}{\strut\hspace{2pt}}\colorbox{white}{\textcolor{black}{##2}}}] at (0,-5) {}; % start year
    }
    \newcommand{\bxwidth}{4.5}
    \newcommand{\bxheight}{2}
    % creates a stretched box as cv entry headline followed by two paragraphs about 
    % the work you did
    % param 1:  event start month/year
    % param 2:  event end month/year
    % param 3:  event name
    % param 4:  level position y
    % param 5:  level position x
    % param 6:  color box
    % param 7:  color text
    % param 8:  level (position, use minus for left placement)
    \newcommand{\cvevent}[8] {
        \foreach \monthf/\yearf in {##1} {
            \foreach \montht/\yeart in {##2} {
                \pgfmathparse{-#3/\fullrange*((\yearf-#1)+(\monthf/12))}
                \let\startexp\pgfmathresult
                \pgfmathparse{-#3/\fullrange*((\yeart-#1)+(\montht/12))}
                \let\endexp\pgfmathresult
                \pgfmathparse{1/(\endexp-\startexp+1)}
                \let\lenexp\pgfmathresult
                \pgfmathparse{0.5*\endexp+0.5*\startexp}
                \let\midexp\pgfmathresult
                
                \pgfmathparse{-#3/\fullrange*((\yearf-#1)+(\monthf/12))}% get the event position
                \let\eventposition\pgfmathresult
                
                \filldraw[fill=##6, draw=none, opacity=0.9] (\startexp ,0.15+##8) rectangle (\endexp, 0.15+##8+0.5);
                
				\draw[draw=##6, line width=1.5pt, anchor=west] (\eventposition,0) -- (\eventposition+##5,##4);% line			
				
				\node[fill=##6,right=-0.1em,inner sep=0.5em, label={[label distance=0]0:\colorbox{##6}{\textcolor{##7}{##3}}}] at (\eventposition+##5,##4) {\textcolor{##7}{##1}};% textbox			
				
				% \node[below, inner sep=1.5em] at (\eventposition,0) {##1};%draw year below

				\draw[fill=white,draw=##6, line width=0.1em]  (\eventposition,0) circle (0.1);%year circle
            }
            \addtocounter{leftcount}{1}
        }
    }

    %--------------------------------------------------------------------------------------
    %   BEGIN
    %--------------------------------------------------------------------------------------

    \begin{tikzpicture}
    \setcounter{leftcount}{1}
    %calc fullrange= number of years
    \pgfmathparse{(#2-#1)}
    \let\fullrange\pgfmathresult
    \draw[draw=#4,line width=4pt] (0,0) -- (-#3,0) ;    %the timeline
    
    %for each year put a horizontal line in place
    \setcounter{yearcount}{1}
    \whiledo{\value{yearcount} < \fullrange}{
        \draw[fill=white,draw=#4, line width=2pt]  (-#3/\fullrange*\value{yearcount},0) circle (0.2);
        \stepcounter{yearcount}
    }
	
	%start year
    \filldraw[fill=white!100,draw=#4,line width=3pt] (0,0) circle (0.5);
    \node[draw=none, label=\textcolor{#4}{\textbf{\small#1}}] at (0,-0.35) {}; 
    %end year
    \filldraw[fill=white!100,draw=#4,line width=5pt] (-#3,0) circle (0.75);
    \node[draw=none, label=\textcolor{#4}{\textbf{\large#2}}] at (-#3,-0.35) {}; 
    
    
}%end begin part of newenv
{\end{tikzpicture}}

%------------------------------------------------------------------ Variablen



\newlength{\rightcolwidth}
\newlength{\leftcolwidth}
\setlength{\leftcolwidth}{0.3\textwidth}
\setlength{\rightcolwidth}{0.65\textwidth}

%------------------------------------------------------------------
\title{Hipster-CV}
\author{\LaTeX{} Ninja}
\date{March 2019}

\pagestyle{empty}
\begin{document}


\thispagestyle{empty}
%-------------------------------------------------------------

\section*{Start}



\header{\bgupper{headerfontbox}{headerfontboxfont}{\bfseries\Huge Achraf Najmi}}{\bg{headerfontbox}{headerfontboxfont}{\large Électro Info Physicien}}{\bg{headerfontbox}{headerfontboxfont}{\small 
\begin{minipage}[t]{0.55\textwidth}
Je suis diplômé en Physique Électronique avec une expérience de travail dans des projets de recherche. 
Je suis toujours à la recherche de nouveaux projets interdisciplinaires passionnants liés à la physique.
\end{minipage}}}{./pic/me.JPG}{headerblue}{3.5cm}{1cm}



%------------------------------------------------

% hier muss die "unsichtbare" Überschrift rein, weil er sonst nicht die Paracols startet... komisch...
\subsection*{}
\vspace{4em}

\setlength{\columnsep}{1.5cm}
\columnratio{0.3}[0.65]
\begin{paracol}{2}
\hbadness5000
%\backgroundcolor{c[1]}[rgb]{1,1,0.8} % cream yellow for column-1 %\backgroundcolor{g}[rgb]{0.8,1,1} % \backgroundcolor{l}[rgb]{0,0,0.7} % dark blue for left margin

\paracolbackgroundoptions

% 0.9,0.9,0.9 -- 0.8,0.8,0.8

\vspace{-2em}
\footnotesize
{\setasidefontcolour
\bgupper{cvyellow}{iconcolour}{Faits} \\
\bg{cvyellow}{iconcolour}{Personnel} \\

\begin{tabular}{ll | l}
\faMale & Nom Complet & Achraf NAJMI \\
\faBirthdayCake & Né le &  24/07/1994 \\
% \faGlobe & Nationality & Moroccan  \\
\faMapMarker & Emplacement & Casablanca, Maroc \\
\end{tabular}

\bigskip

\bg{cvyellow}{iconcolour}{Physique Électronique} \\

\begin{tabular}{l | l}
\faRobot \hspace{0.1em} Électronique & \faChargingStation \hspace{0.1em} Électrotechnique  \\
\faLaptop \hspace{0.1em} Numérique & \faBroadcastTower \hspace{0.1em} Télécommunication \\
\faSitemap  \hspace{0.1em} Réseau
\end{tabular}

\bigskip

\bgupper{cvyellow}{iconcolour}{Capacites} \\
\bg{cvyellow}{iconcolour}{Langues}
\bigskip


\begin{minipage}[t]{\leftcolwidth}
\begin{tabular}{ll | l}
\faLanguage & \textbf{Français} & B2 \pictofraction{\faCircle}{cvpurple}{4}{black!30}{2}{\tiny}\\
\faLanguage & \textbf{Anglais} & A2 \pictofraction{\faCircle}{cvpurple}{2}{black!30}{4}{\tiny}
\end{tabular}
\end{minipage}

\bigskip

\bg{cvyellow}{iconcolour}{Informatique et Programmation}\\

\bubblediagram{{\icon{\faPython}{fontdiagram}{\Large} \\ \textbf{\color{fontdiagram}{Python 3}}}, \faFileCode \hspace{0.1em} pyinstaller, \faSuperscript \hspace{0.1em} sympy, \faSortNumericDown \hspace{0.1em} numpy, \faTable \hspace{0.1em} pandas, \faFeather \hspace{0.1em} tkinter, \faChartLine \hspace{0.1em} matplotlib}


% \bigskip
\smallskip

\bg{cvyellow}{iconcolour}{Environnement}\\

\color{labelcolour}{Syntaxe:}
% \hspace{0.2em} 
\icon{\faPython}{labelcolour}{\Large} 
\color{labelcolour}{3.7 \& 3.8 \hspace{0.7em} OS:}
% \hspace{0.3em} 
\icon{\faWindows}{labelcolour}{\Large} 
% \hspace{0.3em} 
\icon{\faLinux}{labelcolour}{\Large} 

\bigskip
% \smallskip

\bg{cvyellow}{iconcolour}{Programmes Logiciels} \\

\begin{minipage}[t]{0.3\textwidth}
\begin{tabular}{r @{\hspace{0.5em}}l}
     \bg{skilllabelcolour}{iconcolour}{PyCharm IDE} & \barrule{0.4}{0.5em}{materialgreen} \\
     % \bg{skilllabelcolour}{iconcolour}{Microsoft Office} & \barrule{0.4}{0.5em}{materialamber} \\
     \bg{skilllabelcolour}{iconcolour}{Visual Studio IDE} & \barrule{0.35}{0.5em}{materialindigo}\\
     \bg{skilllabelcolour}{iconcolour}{Proteus CAD} & \barrule{0.3}{0.5em}{materialcyan} \\
     \bg{skilllabelcolour}{iconcolour}{\faMarkdown \hspace{0.1em} Markdown} & \barrule{0.25}{0.5em}{materialorange} \\
     \bg{skilllabelcolour}{iconcolour}{\LaTeX} & \barrule{0.2}{0.5em}{materiallime} \\
     \bg{skilllabelcolour}{iconcolour}{\faCodeBranch \hspace{0.1em} \faGit} & \barrule{0.15}{0.5em}{materialteal} \\
    
\end{tabular}


% \dashrule{}{}
\end{minipage}

\smallskip
\bg{cvyellow}{iconcolour}{Technologies}\\
% \parbox[b][110pt][c]{0.30\textwidth}{
% PIE CHART	
\begin{piechart}{360}{1}{bgchart}{iconcolour}{skilllabelcolour}
			\slice{33}{Développement}{cvyellow}
			\slice{28}{Maintenance}{cvred}
			\slice{23}{Schématisation}{headerblue}
			\slice{18}{Désigne}{cvorange}
\end{piechart}\vspace{-4em}
}
\phantom{turn the page}

\phantom{turn the page}
% }
%-----------------------------------------------------------

\switchcolumn

\small
\vspace{-2em}
\section*{Education et Experience}
\begin{timelinehorizontal}{2017}{2022}{11}{black}
	
			\cvevent{2/2017}{9/2018}{Maintenance Informatique}{1}{-4.5}{headerblue}{iconcolour}{-1}
	
			\cvevent{9/2017}{6/2019}{Études de licence : Physique Électronique}{1.6}{-7}{cvyellow}{iconcolour}{-0.7}
			
			\cvevent{9/2019}{3/2021}{Développement Informatique : \hspace{0.1em} \faPython \hspace{0.1em} Python 3 : Syntaxe, GUI, Data Science}{2.2}{-5.5}{cvyellow}{iconcolour}{-0.7}
			
\end{timelinehorizontal}

% \vspace{-0.5em}

% \begin{minipage}[t]{0.4\textwidth}
\section*{Diplômes et Certifications}
\begin{tabular}{r| p{0.45\textwidth} c}

    \cvcertpy{2021}{Developer Certification \color{cvred}}{Scientific Computing with Python}{freeCodeCamp \color{headerblue}}{\href{https://freecodecamp.org/certification/osirishoruseye/scientific-computing-with-python-v7}{\textcolor{black!70}{\faFreeCodeCamp} \hspace{1pt} \textcolor{black!70}{freecodecamp.org/certification/osirishoruseye/scientific-computing-with-python-v7}}}{./pic/python.png} \\

    \cvdegree{2019}{Licence d'Études Fondamentales (LEF) \color{cvred}}{Université Hassan II de Casablanca}{FSBM \color{headerblue}}{Filière : Science Matière Physique | Parcours : Électronique}{./pic/univh2fsbm.png} \\
    
    \cvdegree{2017}{Diplôme d'Études Universitaires Générales (DEUG) \color{cvred}}{Université Hassan II de Casablanca}{FSBM \color{headerblue}}{Filière : Science Matière Physique}{./pic/univh2fsbm.png} \\
    
    % \cvschool{2012}{Diplôme du Baccalauréat \color{cvred}}{Lycée}{Ibno Zaidone \color{headerblue}}{Série : Sciences Expérimentales | Option : Science de la Vie et la Terre}{./pic/minister.jpg}
\end{tabular}
% \end{minipage}\hfill

% \vspace{-0.5em}
\section*{Projets}
\begin{tabular}{r| p{0.45\textwidth} c}
    \cvpython{2021}{Projet de Hydrogéologie}{Python 3.7}{Hydrologie des eaux souterraines \color{cvyellow}}{\href{https://github.com/OsirisHorusEye/Hydrogeologie}{\icon{\faGithub}{cvpurple}{}\textcolor{black!70}{github.com/OsirisHorusEye/Hydrogeologie (PROJET PRIVÉ)}}}{./pic/earth.png} \\
    
    \cvpython{2020}{Projet de MathPy}{Python 3.7}{Calculatrice Scientifique Moderne \color{cvyellow}}{\href{https://www.github.com/OsirisHorusEye/MathPy}{\icon{\faGithub}{cvpurple}{}\textcolor{black!70}{github.com/OsirisHorusEye/MathPy (PROJET PRIVÉ)}}}{./pic/MathPy.png} \\
    
    \cvproject{2018}{Projet de Fin d'Études}{UHIIC}{Faculté des Sciences Ben M'sick \color{cvyellow}}{Conception d'un système photovoltaïque et réalisation d'un suiveur solaire}{./pic/univh2fsbm.png} \\
    
    % \cvdegree{2018}{Projet Entreprenariat}{UHIIC}{Faculté des Sciences Ben M'sick \color{cvorange}}{Projet de Création une Entreprise Commerciale "electrocherry".}{./pic/electro.png}
\end{tabular}

\begin{minipage}[t]{0.3\textwidth}
\vspace{1em}
\section*{Info et Program}
\textcolor{black}{Actuellement, je développe des applications en temps réel avec Python, et crée une interface graphique avec tkinter, et extrait du script `file.py` un fichier exécutable par pyinstaller en utilisant auto py to exe.}

\end{minipage}\hfill
\begin{minipage}[t]{0.3\textwidth}
\vspace{1em}
\section*{Programmes Logiciels}
\textcolor{black}{J'utilise régulièrement pour mes projets de développement PyCharm Community et Visual Studio comme IDE, et j'utilise pour la capture schématique et la simulation Proteus CAD.}

\end{minipage}

\begin{minipage}[t]{0.65\textwidth}
\vspace{1em}
\section*{Activites}
\mbox{
	\parbox[b][2cm][c]{7cm}{
		\progressarc{0.55em}{cvyellow}{0.6cm}{timelinebg}{\huge\color{black}\faPlayCircle}{Tutoriels \\}{below}{82.5} \hfill
		\hspace{-1em}
		\progressarc{0.55em}{cvyellow}{0.6cm}{timelinebg}{\huge\color{black}\faNewspaper}{Actualités \\Technologies}{below}{75} \hfill
		\hspace{-1em}
		\progressarc{0.55em}{cvyellow}{0.6cm}{timelinebg}{\huge\color{black}\faChess}{Jeux \\Cérébrale}{below}{67.5} \hfill
		\progressarc{0.55em}{cvyellow}{0.6cm}{timelinebg}{\huge\color{black}\faBicycle}{Sport \\}{below}{60}
	}

	\parbox[b][2.2cm][c]{5cm}{
			\textcolor{black}{Parmis les dominantes activités que je pratique est de naviguer sur internet à la recherche de nouvelles sur le monde d'informatique et  entraîné sur tous les tutoriels qui m'intéresse.}
		}
	}

\end{minipage}









\vfill{} % Whitespace before final footer

%----------------------------------------------------------------------------------------
%	FINAL FOOTER
%----------------------------------------------------------------------------------------
\setlength{\parindent}{0pt}
\begin{minipage}[t]{\rightcolwidth}
\begin{center}\fontfamily{\sfdefault}\selectfont \color{black!70}
{\href{callto:+212622902221}{\icon{\faMobile}{black}{} \textcolor{black!70}{+212 6 22 90 22 21}} \href{mailto:najmi.achraf@gmail.com}{\icon{\faAt}{red}{} \textcolor{black!70}{najmi.achraf@gmail.com}} \href{https://www.github.com/OsirisHorusEye}{\icon{\faGithub}{cvpurple}{} \textcolor{black!70}{github.com/OsirisHorusEye}}  \newline\href{https://www.linkedin.com/in/osirishoruseye}{\icon{\faLinkedin}{blue}{} \textcolor{black!70}{linkedin.com/in/osirishoruseye}}\href{https://stackoverflow.com/users/12854948}{\icon{\faStackOverflow}{black}{} \textcolor{black!70}{stackoverflow.com/users/12854948}}
}
\end{center}\vspace{-2em}
\end{minipage}


\end{paracol}
\end{document}
